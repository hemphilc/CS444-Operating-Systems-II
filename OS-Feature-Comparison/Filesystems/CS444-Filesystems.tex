\documentclass[letterpaper,10pt,draftclsnofoot,onecolumn]{IEEEtran}
\usepackage{graphicx, geometry, hyperref, geometry, listings, enumitem, balance, longtable, url, color, float, alltt, amsthm, amsmath, amssymb, multirow, setspace, courier}
\include{pygments.tex}
%\usepackage{minted}
\geometry{margin=.75in}

\definecolor{codegreen}{rgb}{0,0.6,0}
\definecolor{codegray}{rgb}{0.5,0.5,0.5}
\definecolor{codepurple}{rgb}{0.58,0,0.82}
\definecolor{backcolour}{rgb}{1.0,1.0,1.0}

\usepackage{listings}
\lstset{
         basicstyle=\footnotesize\ttfamily,
         backgroundcolor=\color{backcolour},
    	 commentstyle=\color{codegreen},
    	 keywordstyle=\color{magenta},
    	 numberstyle=\tiny\color{codegray},
    	 stringstyle=\color{codepurple},
         morecomment=[l][\color{magenta}]{\#},
         numberstyle=\tiny,
         numbersep=5pt,
         numbers=left,
         tabsize=1,                
         extendedchars=true,      
         breaklines=true,        
         showspaces=false,      
         showtabs=false,       
         xleftmargin=17pt,
         framexleftmargin=17pt,
         framexrightmargin=5pt,
         framexbottommargin=4pt,
         showstringspaces=false 
 }

\def \Author{Corey Hemphill}
\def \Title{Operating System Feature Comparison}
\def \Subtitle{Filesystems}
\def \Term{cs444 Fall 2017}
\def \DueDate{December 7, 2017}

\def \DocType{
	Operating Systems II
}
			
\newcommand{\NameSigPair}[1]{\par
\makebox[2.75in][r]{#1} \hfil 	
\makebox[3.25in]{\makebox[2.25in]{\hrulefill} \hfill
\makebox[.75in]{\hrulefill}}
\par\vspace{-12pt} \textit{\tiny\noindent
\makebox[2.75in]{} \hfil
\makebox[3.25in]{\makebox[2.25in][r]{Signature} \hfill
\makebox[.75in][r]{Date}}}}

\begin{document}
\begin{titlepage}
    \pagenumbering{gobble}
    \begin{singlespace}
        \hfill  
        \par\vspace{.2in}
        \centering
        \scshape {
            \huge  \DocType \par
           	\huge \Term \par
            {\large \DueDate}\par
            \vspace{.5in}
            \textbf{\Huge \Title}\par
            {\large \Subtitle}\par
            \vspace{.5in}          
            {\large By }\par
           	\textbf{\Author}\par
            \vspace{5pt}
            }
            \vspace{120pt}
        
        \begin{abstract}
        This document examines, compares, and contrasts operating system filesystems implementations in Windows, FreeBSD, and Linux operating systems.
        \end{abstract}
        
    \end{singlespace}
\end{titlepage}
\newpage

\section{Compare and Contrast FreeBSD and Linux Filesystems}
\noindent  The FreeBSD operating system kernel supports both UFS and ZFS filesystems. Filesystems are a fundamental component to all modern operating systems, and many operating systems vary in which file system they support by default. Unix File System (UFS), as the same implies, is a file system that is used by many UNIX-like operating systems, and FreeBSD is no exception. A UFS disk volume has a number of key features; at the start of the volume’s partition, there is a reserved space for the system’s boot file blocks, followed by a superblock which contains several different important file system parameters. This disk volume is comprised of a number of cylinder groups which each contain a copy of the aforementioned superblock, nodes, and data blocks. The UFS is also known as the Fast File System (FFS). Given that the FreeBSD kernel is tuned in favor of performance, it makes sense that it uses a file system designed to be fast. The ZFS file system can also be used natively in FreeBSD. ZFS is a unique file system in that it behaves as both the file system and as a system volume manager. The primary perks of ZFS include data integrity, pooled storage, and performance. Overall, it’s easy to see why these two filesystems work exceptionally well within FreeBSD systems \cite{FreeBSD1FS} \cite{FreeBSD2FS} \cite{FreeBSD3FS}.\\

\noindent Remember that in Linux and FreeBSD, everything is a file, and if it’s not a file, it’s a directory, or a process of some kind. So, as expected, Linux and FreeBSD utilize similar filesystems since both operating systems are descendants of UNIX.  FreeBSD also natively supports a number of different Linux filesystems. Linux also natively supports several different filesystems. In fact, many Linux filesystems are commonly used in many modern day operating systems. Some of the Linux filesystems include ext2, ext3, JFS, and XFS. All of these filesystems have trade-offs in terms of speed, overall efficiency, and other various performance metrics. These metrics should be greatly considered when determining which of these filesystems to use for solving a given problem \cite{Linux1FS} \cite{Linux2FS}.\\

\section{Compare and Contrast Windows and Linux Filesystems}
\noindent The Windows operating system utilizes the New Technology File System (NTFS). The primary design goals of NTFS are compatibility, reliability, and performance. NTFS is particularly fast at performing read, write, and search operations. When formatting an NTFS volume, a number of metadata files are created which contain data pertaining to all of the files that reside on a given volume. With NTFS, everything that lives on a disk volume is considered a file, and everything that exists within a given file is called an attribute. The NTFS filesystem supports a number of security features which includes file and folder permissions, and disk encryption. Windows also supports File Allocation Table (FAT) filesystems, which are inherited from the DOS days. FAT filesystem volumes are very simple in design, and they consist of a boot sector, a block allocation table, and storage space for storing folders and files \cite{MSWindows1FS} \cite{MSWindows2FS}.\\

\noindent Linux and Windows have a number of notable difference between them in regard to filesystems. When it comes to NTFS, Linux can read and write to NTFS, but cannot boot from NTFS. Windows does not natively support Linux filesystems, or any other filesystems really, other than FAT and NTFS. Linux, in this case is far more versatile than Windows. There are, however, ways to read and write to non-native filesystems in Windows through some kind of third-party application, but whether or not that is advised is questionable \cite{Linux1FS} \cite{Linux2FS}.\\

\section{Section Conclusion}
\noindent When it comes to filesystems, FreeBSD and Linux share a lot in common in that they support many of the same filesystems. This is surprising to no one. Also, not surprising is that Windows does not support any of the Linux based filesystems, and Linux supports all of the Windows based filesystems. In Windows, it seems, you only get what you pay for. Linux users benefit from the fact that the project is open-source, and you get what other people needed, and implemented, at some point in time. This helps to present Linux as a more well-rounded operating system than Windows.

\newpage
\bibliographystyle{IEEEtran}
\bibliography{CS444-Filesystems}
\end{document}
