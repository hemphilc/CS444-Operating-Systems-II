\documentclass[onecolumn, draftclsnofoot,10pt, compsoc]{IEEEtran}
\usepackage{graphicx}
\usepackage{url}
\usepackage{setspace}
\usepackage{multirow}

\usepackage{geometry}
\geometry{textheight=9.5in, textwidth=7in}

% 1. Fill in these details
\def \OperatingSystemTwo{Group 46}
\def \OSTwoGroupNumber{46}
\def \GroupMemberOne{Corey Hemphill}
\def \GroupMemberTwo{Jason Ye}
\def \HomeworkAssignmentOne{I/O Elevators}
\def \HomeworkDueDate{October 23, 2017}


% 2. Uncomment the appropriate line below so that the document type works
\def \DocType{	%Problem Statement
				%Requirements Document
				%Technology Review
				%Design Document
				Project 2 Report
				}
			
\newcommand{\NameSigPair}[1]{\par
\makebox[2.75in][r]{#1} \hfil 	\makebox[3.25in]{\makebox[2.25in]{\hrulefill} \hfill	\makebox[.75in]{\hrulefill}}
\par\vspace{-12pt} \textit{\tiny\noindent
\makebox[2.75in]{} \hfil		\makebox[3.25in]{\makebox[2.25in][r]{Signature} \hfill	\makebox[.75in][r]{Date}}}}
% 3. If the document is not to be signed, uncomment the RENEWcommand below
%\renewcommand{\NameSigPair}[1]{#1}

%%%%%%%%%%%%%%%%%%%%%%%%%%%%%%%%%%%%%%%
\begin{document}
\begin{titlepage}
    \pagenumbering{gobble}
    \begin{singlespace}
    	%\includegraphics[height=4cm]{coe_v_spot1}
        \hfill 
        % 4. If you have a logo, use this include graphics command to put it on the coversheet.
        %\includegraphics[height=4cm]{CompanyLogo}   
        \par\vspace{.2in}
        \centering
        \scshape{
            \huge  \DocType \par
           	\huge cs444 Fall17 \par
            {\large\today}\par
            \vspace{.5in}
            \textbf{\Huge\HomeworkAssignmentOne}\par
            \vspace{.5in}
           
            {\large Prepared by }\par
           	\textbf{\GroupMemberOne}\par
            \textbf{\GroupMemberTwo}\par
            % 5. comment out the line below this one if you do not wish to name your team
   
            \vspace{5pt}
            
            \textbf{\Huge\ \OperatingSystemTwo}\par
            }
            \vspace{60pt}
        
        \begin{abstract}
        % 6. Fill in your abstract 
        This project report is a summary of Homework 2 for Operating Systems II at Oregon State University, Fall 2017. This document includes a list of commands used to complete necessary tasks, a list of Qemu command flags used, a summary of the producer-consumer concurrency problem implementation, a version control log for homework files, and a comprehensive team member work log/history.
        \end{abstract} 
        
    \end{singlespace}
\end{titlepage}
\newpage
\pagenumbering{arabic}
\tableofcontents
% 7. uncomment this (if applicable). Consider adding a page break.
%\listoffigures
%\listoftables
\clearpage

% 8. now you write!

\section{Design Plan}
\noindent
First, we did a lot of research on C-LOOK I/O scheduler and tried understand its operation and functionalities. Then we look into the block directory for Noop scheduler and use it as our guidance for creating sstf-iosched.c file. Using the provided file, we simply modify the two main functions: Dispatcher and add request. Based on our understanding, the algorithm maintains a request queue sorted in the order that they would be dispatched, and when the head dispatches the requests, it will travel up to the highest request then come back down to the restart dispatching. 
\\

\section{Answer to Each Questions }
\subsection{What do you think the main point of this assignment is}

\noindent 
The main focus of this assignment was to learn about I/O Scheduler and modifying the Linux kernel. It seems to us that this assignment prepares us to look deeper and understand more about kernel implementation. Also, we got to learn more about C-LOOK scheduler.  
\\

\subsection{How did you personally approach the problem? Design decisions, algorithm, etc.}

\noindent 
First, we started out examining the noop-iosched.c file provided in the block directory. We modified the file based on our understanding and research on I/O scheduling. We used majority of the implementation from noop-iosched.c and only changed the functions of adding request to queue and dispatching request. 
\\

\subsection{How did you ensure your solution was correct? Testing details, for instance.}
\noindent 
To test our program, we decided on making a bash script that read files from the directories and output a file. It basically read/write two different files: TESTFILE and TESTFILE2. Also, we implemented a DEBUG variable and print statements to ensure our functions are working properly.  
\\

\subsection{What did you learn?}
\noindent 
We learned how to implement C-LOOK I/O scheduling algorithm and how to interact with it on the VM. We learned how to use data structures and list in the kernel. Also, we get to learn how to build patches and running it in a VM using Qemu. \\

\noindent 
"More about kernel" 
\\

\section{Version Control Log}

\begin{tabular}{l l l}

\textbf{Detail SHA} & \textbf{Author} & \textbf{Description}\\
\hline {a0c57c07671e72523099815d5273c80ce20bcc70} & hemphilc & committed project 2 folder and .c file  \\
\hline {ac9a66ec1148b5ee2ec486e1c1f1781b24c70e45} & hemphilc & committed make, IEETran, README, and .tex files \\
\hline {45d131ed1089ae207f7eae1a34b273c1c915e030} & hemphilc & updated README \\
\hline {0eff62509f5a40b062bbd59227a8a4b2121d270e} & {yeja} & cloned .c file and added prototype functions. \\
\hline {a6e77b6ea7c9d9c65c3f2f94282ac49963ca8137} & {yeja} & Updated sstfiosched.c file \\
\hline {ecb3f61faf0e2e0250718cb7ad73abf3c13dfbca} & {yeja} & updated and started modifying add and dispatcher functions \\
\hline {dc284b3e9bf231ff1a78210be8dc48f435917cc2} & {hemphilc} & Updated README \\
\hline {964f84a823856c62f72818d201e78b2f73198882} & {hemphilc} & v.1.0 of sstfiosched.c \\
\hline {52b3d6796edf406b610ebbbe4456074a2c99017d} & {hemphilc} & v.1.0.0 of sstfiosched.c \\
\hline {74c23cd1c7a7e3d1618e25b993c9a11487d3b3ba} & {hemphilc} & Created blkdev.h \\
\hline {e434f64aacbf187bf329bf62b6e04a74afc781ab} & {hemphilc} & Updated .tex file \\
\hline {da36cf66aae9e7c601bfcfe483be3c496e761535} & {yeja} & v.1.1.0, Updated sstfiosched.c \\
\hline {ce820f064d26765907f3f698897e25ed38cd76e9} & {yeja} & Updated sstfiosched.c \\
\hline {d34989203dcd9eaf93d65ddb09d3f5a701e31dab} & {hemphilc} & Delete blkdev.h \\
\hline {c51271c5b39c5b44ff4c07b173f84e5b59115242} & {hemphilc} & Update README
 \\
\hline {5f4bfc80ba25c2c5bcfa65403c634ab423251523} & {hemphilc} & Update sstf-iosched.c \\
\hline {fe24cddb1f35416116cd68e911a9f6e763127805} & {hemphilc} & Update sstf-iosched.c \\
\hline {ec95a6bae6809a98537a3496d4092143094a73c7} & {hemphilc} & Update .tex file\\
\hline {cc32a6817818666afbe6091341b28654166684e6} & {hemphilc} & v.1.0.1 \\

\end{tabular}

\section{Work Log}
% List out the work by each member. Briefly state the date and description of the work. 
\begin{itemize}

\item \textit{10/15/2017}\\ Corey created a project 2 folder, made README, .c and .tex files inside the folder. \\
\item \textit{10/17/2017}\\ Jason and Corey doing research on C-LOOK I/O Scheduler. \\
\item \textit{10/18/2017}\\ Jason started to create functions that might work in sstf-iosched.c. \\ 
\item \textit{10/20/2017}\\ Corey updated the README file and copied the functions from noo-iosched.c to sstf-iosched.c. \\
\item \textit{10/21/2017}\\ Jason started to modify the .c file and worked on dispatcher and addrequest functions. \\
\item \textit{10/21/2017}\\ Corey updated and debug the .c file. Also, he started to do research on patching in kernel. \\ 
\item \textit{10/22/2017}\\ Corey updated .tex and started adding more print statement into the .c file for testing purposes. \\
\item \textit{10/22/2017}\\
\end{itemize}

\end{document}
